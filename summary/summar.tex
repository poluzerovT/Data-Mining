\documentclass[a4paper, 12pt]{article}

\usepackage{cmap}					% поиск в PDF
\usepackage[T2A]{fontenc}			% кодировка
\usepackage[utf8]{inputenc}			% кодировка исходного текста
\usepackage[english,russian]{babel}	% локализация и переносы


\begin{document}
    
    \section{Понятия}

    $\varphi : X \rightarrow {0, 1}$ - предикат на множестве объектов.
    Предикат $\varphi$ покрывает объект $x$, если $\varphi(x) = 1$

    Предикат называют \textbf{закономерностью}, если он покрывает достаточно
    много объектов класса $c$ и при этом не слишком много объектов других классов.

    Закономерности которые описываются простой логической формулой называют
    правилами (rule). Процесс поиска правил по выборке называют 
    \textbf{поиском знаний из данных} (knowledge discovery).
    Алгоритмы объединяющие несколько правил называются \textbf{логическими
    алгоритмами классификации}.
    
    \section{Информативность}

    \subsection{Эвристическое определение информативности}

    $D = (X, Y), |D| = \ell$

    $P$ - число объектов класса $c$ в выборке $D$

    $p(\varphi)$ - объекты из $P$, которые покрывает $\varphi$

    $N$ - число объектов класса не $c$

    $n(\varphi)$ - объекты из $N$, которые покрывает $\varphi$

    $P + N = \ell$

    Требуется построить информативный предикат $\varphi$, который одновременно
    $p(\varphi) \rightarrow \max$ и $n(\varphi) \rightarrow \min$.

    Доля негативных среди всех выделяемых объектов
    \[
    E(\varphi, D) = \frac{n}{p + n}
    \]

    доля выделяемых позитивных
    \[
        D(\varphi, D) = \frac{p}{\ell}
    \]

    Определение.
    Предикат $\varphi$ называется логической $\varepsilon, \delta$-закономерностью
    для класса $c$, если $E \le \varepsilon$ и $D \ge \delta$ при заданных достаточно
    малом $\varepsilon$ и достаточно большом $\delta$ из отрезка $[0, 1]$.

    \subsection{Статистическая инфомративность}

     Пусть $X$ - вероятностное пространство.
    Справедлива гипотеза о независимости событий ${x: y^*(x) = c}$ и ${x: \varphi(x) = 1}$
    Тогда вероятность реализации пары $(p, n)$ распределена по гипергеометрическому закону

    \[
    h(p, n) = \frac{C_P^p C_N^n}{C_{P + N}^{p + n}}
    \]

    Чем меньше вероятность пары $(p, n)$, тем более значимой является связь между 
    $y^*$ и $\varphi$. Другими словами, если реализовалось маловероятное событие, то,
    скорее всео, оно не случайно, а закономерно.

    Определение.
    Информативность предиката $\varphi(x)$ относительно класса $c$ по выборке $D$
    есть
    \[
    I(\varphi, D) = - \ln h(p, n)
    \]

    Предикат $\varphi(x)$ будем назвать статистической закономерностью для класса $c$, если
    $I(\varphi, D) \ge I_0$ при заданном достаточно большом $I_0$.
    
    \subsection{Энтропийная инфомративность}

    Если имеется два исхода $\omega_1$ и $\omega_2$ с вероятностями
    $q_0$ и $q_1 = 1 - q_0$, то количество информации, связанное с исходом
    $\omega_i$ по определению равно $-\log_2 q_i$.

    Энтропия, определяемая как матожидание количества информации:
    \[
    H(q_0, q_1) = - q_0 \log_2 q_0 - q_1 \log_2 q_1
    \]

    Будем считать появление объекта $c$ исходом $\omega_0$, а появление объекта
    любого другого класса исходом $\omega_1$. Тогда можно вычислить энтропию выборки
    \[
    \hat{H}(P, N) = H \left(\frac{P}{P + N}, \frac{N}{P + N} \right)
    \]

    Допустим предикат $\varphi$ выдели $p$ объектов из $P$ и $n$ объектов из $N$.
    Тогда энтропия подвыборки ${x \in X | \varphi(x) = 1}$ есть $\hat{H}(p, n)$.

    вероятностность появления объекта из этой выборки оценивается как $(p + n) / (P + N)$.

    Аналогично для подвыборки ${x \in X | \varphi(x) =0}$, энтропия равна $\hat{H}(P-p, N-n)$,
    а вероятность появления объекта из неё оценивается как $(P-p + N-n) / (P + N)$.
    Таким образом, энтропия всей выборки после получения информации $\varphi$ становится равна
    \[
    \hat{H}_{\varphi}(p, n) = \frac{p + n}{P + N}\hat{H}(p, n) 
    + \frac{P + N - p - n}{P + N} \hat{H}(P - p, N - n)
    \]

    Уменьшение энтропии составляет
    \[
    IGain(\varphi, D) = \hat{H} - \hat{H}_{\varphi}
    \]

    Так же это называют информационным выигрышем --- количество информации об исходном
    делении выборки на два класса <<$c$>> и <<не $c$>>, которое содердится в предикате
    $\varphi$.

    Определение.
    Предикат $\varphi$ является закономерностью по энтропийному критенрию информативности,
    есои $IGain(\varphi, D) > G_o$ при достаточно большом $G_0$.

    Теорема.
    Энтропийный критерий $IGain$ асимптотически эквивалентен статистическому $I$
    \[
    IGain(\varphi, D) \rightarrow_{\ell \rightarrow +\infty} \frac{1}{\ell \log_2} I(\varphi, D)
    \]
\end{document}